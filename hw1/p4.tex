\begin{question}{4 (15') (More on Tutte Matrix)} 
In this problem, we consider Tutte Matrix and matching in more general problems.
\begin{itemize}
	\item[a. (7')] Given a \textit{bipartite graph} $G = (U, V, E)$, define its Tutte Matrix $A(G)$ as 
\begin{equation*}
	A(G)_{ij}:=\begin{cases}
	x_{ij} & \text{if }(u_i,v_j)\in E;\\
	0 & \text{otherwise.}
	\end{cases}
\end{equation*}
Show that the maximum size of a matching in $G$ is exactly equal to the rank of $A(G)$.
	\item[b. (8')] Given a \textit{simple graph} $G=(V,E)$, define its Tutte matrix $B(G)$ as
\begin{equation*}
	B(G)_{ij}:=\begin{cases}
	x_{ij} & \text{if }(v_i,v_j)\in E\text{ and }i<j;\\
	-x_{ji} & \text{if }(v_i,v_j)\in E\text{ and }i>j;\\
	0 & \text{otherwise.}
	\end{cases}
\end{equation*}
Show that $G$ has a perfect matching $\Leftrightarrow\det B(G)\not\equiv 0$. 

\textit{[Hint: You can find some hints on the lecture notes. ]}

\ta{This actually gives a $O(n ^ 3)$ algorithm for solving general matching on a graph with $n$ nodes. You're encouraged to find it out. }
\end{itemize}
\end{question}  

\begin{answer}
	\begin{enumerate}[label=\alph*).]
		\item Notice that each possible matching $(I, J, E_{I,J})$ in $G$ corresponds to a minor of $A(G)$, denoted as $A(G)_{I,J}$ where $I\subseteq U, J\subseteq V$ and $|I|=|J|$. 
		Apply the similar argument in the notes to $A(G)_{I,J}$, we have 
		\begin{align*}
			\det A(G)_{I,J} \not \equiv 0 \quad \text{ iff } \quad \text{$(I, J, E_{I,J})$ contains a perfect matching.}
		\end{align*}
		Suppose the maximum size of a matching in $G$ is $k$, then there exists a set of $k$ rows and $k$ columns in $A(G)$ such that the corresponding minor is non-zero, and any minor of $G$ with size $k+1$ is zero. Therefore, the rank of $A(G)$ is exactly equal to the maximum size of a matching in $G$.
		\item Let $m = |V|$. Consider the matrix $B(G)$, we have
		\begin{align}
			\label{eq:det}
			B(G) = \sum_{\sigma} \text{sgn}(\sigma) \prod_{i=1}^{n} B(G)_{i, \sigma(i)}. 
		\end{align}
		Notice that each non-zero term in the summation \eqref{eq:det} corresponds to a \textit{directed cycle cover} in $G$, respectively.
		For example, suppose $G$ has one directed cycle cover with $k$ cycles, denote as $\{(v_{j_1}^i \to v_{j_2}^i \to \cdots \to v_{j_{r_i}}^i \to v_{j_1}^i): i\in \{1,\cdots,k\}\}$ where the subscript $j_t$ denotes the index of the vertex in $V$.
		Then the corresponding permutation is:    
		\[
			\text{for } i \in \{1, \cdots, k\}:\quad \sigma(j_t) = j_{t+1}, \forall t\in \{1,\cdots,r_i-1\}, \sigma(j_{r_i}) = j_1. 
		\]
		\underline{We first prove that $G$ has a perfect matching $\implies \det B(G) \not\equiv 0$}. 
		Assume the perfect matching of $G$ is $\{(v_{2k-1}, v_{2k})\}_{k=1}^{m/2}$. 
		Then the directed cycle cover is $\{(v_{2k-1} \to v_{2k} \to v_{2k-1})\}_{k=1}^{m/2}$, 
		and the corresponding term is 
		\begin{align}
			\label{eq:cycle}
			\text{sgn}(\sigma)\cdot \prod_{i=1}^{m/2} \left(-x_{2k-1, 2k}^2\right) \not\equiv 0.
		\end{align}		
		Furthermore, this term cannot be canceled by any other term 
		beacuse any other non-zero term which corresponds to a cycle cover 
		must contain at least one variable that is not in \eqref{eq:cycle}.
		
		Therefore, $\det B(G) \not\equiv 0$.

		\underline{Now we prove that $\det B(G) \not\equiv 0 \implies G$ has a perfect matching}.
		Consider any permutation $\sigma_1$ containing a directed cycle with \textit{odd} length. 
		Let $\sigma_2$ be the permutation obtained by reversing the direction of this cycle in $\sigma_1$ (keeping other cycles unchanged).
		Then we try to prove that \textbf{Claim: the corresponding terms will cancel each other in $\det B(G)$}.

		Let the odd length cycle of $\sigma_1$ be $(v_{j_1}\to \cdots \to v_{j_{g}} \to v_{j_1})$ where $g$ is odd. Notice that $\text{sgn}(\sigma_1) = \text{sgn}(\sigma_2)$, and beacuse $B(G)_{i,j} = -B(G)_{j,i}$, we have
		\begin{align*}
			\prod_{i=1}^{g} B(G)_{j_i, j_{i+1}} = x_{j_g, j_1}\prod_{i=1}^{g-1} x_{j_i, j_{i+1}} = -x_{j_1, j_g}\prod_{i=1}^{g-1} x_{j_{g-i+1}, j_{g-i}}.
			= -\prod_{i=1}^{g} B(G)_{j_{g-i+1}, j_{g-i}}.
		\end{align*}
		Therefore, the \textbf{Claim} holds and any permutation contains odd length cycle will be canceled by its reverse permutation in $\det B(G)$.
		
		Then we can conclude that the non-zero contributions in $\det B(G)$ are all from the permutations containing only \textit{even} length cycle covers.
		Furthermore, we can easily construct a perfect matching from such even-length cycle cover by removing half of the edges in each cycle like the following figure:

		\begin{figure}[htbp]
			\centering
			\begin{tikzpicture}
				\GraphInit[vstyle=Normal]
				\SetGraphUnit{2}
				\Vertices{circle}{1,2,3,4,5,6}
				\SetUpEdge[style={->,bend right, thick},
				color=red]
				\Edge(1)(2)
				\Edge(3)(4)
				\Edge(5)(6)
				\SetUpEdge[style={->,bend right, thick},
				color=blue]
				\Edge(2)(3)
				\Edge(4)(5)
				\Edge(6)(1)
			\end{tikzpicture}
			\caption{Construct a perfect matching by removing red edges.}
		\end{figure}

		Therefore, $G$ has a perfect matching $\iff \det B(G) \not\equiv 0$.
	\end{enumerate}
	\ed
\end{answer}