\begin{question}{3 (20') (Schwartz--Zippel is More General)} 
    Our course begins with two examples of randomized algorithms: checking matrix multiplication and checking associativity. It turns out that both can be tackled using Schwartz--Zippel's algorithm. In this problem, you will give equal or better error bounds by constructing polynomials for the two problems.
 
	\begin{itemize}
		\item[a. (10')] Given a binary operator $\circ$ on a set $X$ of size $n$, we wish to decide if $\circ$ is associative. In our class, we defined operator $\circ$ on $2^{X}$ to check the associativity on $X=\{x_1,\cdots,x_n\}$. Similarly we can define operator $\circ$ on $p^X$ where $p\geq 5$ is a given prime. In other words, every element in $p^X$ are in the form of $\sum_{i=1}^{n}r_ix_i$ with $r_i\in\mathbb{F}_p$. 
		
		Base on this, please design a randomized algorithm similar to the one in class. However, this time, you may want to analyze its error bound by applying Schwartz--Zippel Lemma to a polynomial, rather than inclusive-exclusive argument. (You do not need to prove the equivalence to associativity. Schwartz--Zippel Lemma also works for finite fields, provided the size of field is larger than the degree of polynomial --- you do not need to prove this, either. )
		
		\textit{[Hint: Construct multivariate polynomials $F_1(\cdot),\cdots,F_n(\cdot)$ such that $\sum_{i=1}^{n}F_i(\cdot)x_i\equiv 0$ iff $\circ$ is associative over $p^X$. ]}
		
		\item [b. (10')] Design a randomized algorithm that finds a counterexample when $\circ$ is not associative over $X$. Analyze the time complexity of your algorithm.
	\end{itemize}
\end{question} 

\begin{answer}
	\begin{enumerate}[label=\alph*).]
		\item The algorithm is as follows:
		\begin{algo}
			\caption{\textbf{Check Associativity}}
			\label{algo:check}
			\begin{algorithmic}[1]
				\Require A binary operator $\circ$ on a set $X$ of size $n$.
				\Ensure Whether $\circ$ is associative over $p^X$
				\State Randomly sample $\{r_i\}_{i=1}^n, \{s_j\}_{j=1}^n, \{t_k\}_{k=1}^n$ from $\mathbb{F}_p$ independently and uniformly.
				\State $R \leftarrow \sum_{i=1}^{n}r_ix_i, S \leftarrow \sum_{j=1}^{n}s_jx_j, T \leftarrow \sum_{k=1}^{n}t_kx_k$.
				\If{$(R\circ S)\circ T\neq R\circ(S\circ T)$}
					\State \Return \text{False}
				\Else 
					\State \Return \text{True}
				\EndIf
			\end{algorithmic}
		\end{algo}
		Let $F(x) = (R\circ S)\circ T - R\circ(S\circ T)$, and denote $F_i(\cdot)$ as the coefficient of $x_i$ in $F(x)$. Then we have
			$F(x) = \sum_{i=1}^{n}F_i(\cdot)x_i. $
		And it's easy to verify that $F(x) \equiv 0$ iff $\circ$ is associative over $p^X$. Note that $\deg(F(x)) \le 3$. 
		By Schwartz-Zippel Lemma, the error bound of this algorithm is 
		\begin{align*}
			\Pr[R\circ S\circ T = R\circ S\circ T|\text{$\circ$ is not associative}] \le \frac{3}{|\mathbb{F}_p|} = \frac{3}{p}.
		\end{align*}
		\item We first repeat \textbf{Algorithm 1} until finding $R, S, T$ such that $(R\circ S)\circ T\neq R\circ(S\circ T)$. 
		And then find the counterexample $r_i, s_j, t_k \in X$ by the following algorithm:
		\begin{algo}
			\caption{\textbf{Find Counterexample}}
			\label{algo:find}
			\begin{algorithmic}[1]
				\Require A binary operator $\circ$ on a set $X$ of size $n$.
				\Ensure A counterexample $(r_i, s_j, t_k)$ such that $(r_i\circ s_j)\circ t_k\neq r_i\circ(s_j\circ t_k)$.
				\State Repeat \textbf{Algorithm \ref{algo:check}} until finding $R, S, T$ such that $(R\circ S)\circ T\neq R\circ(S\circ T)$.
				\State Denote $R = \sum_{i=1}^{n}r_ix_i, S = \sum_{j=1}^{n}s_jx_j, T = \sum_{k=1}^{n}t_kx_k$.
				\While{$\max\{|R|, |S|, |T|\} > 1$} \Comment{$|R| := \#\{r_i|r_i \neq 0\}$, etc.}
					\State Split $S = \sum_{i=1}^{\lfloor n/2\rfloor}s_ix_i + \sum_{i=\lfloor n/2\rfloor+1}^{n}s_ix_i := S_1 + S_2$. \Comment {assume $|S| = \max\{|R|, |S|, |T|\}$}
					\If{$(R\circ S_1)\circ T = R\circ(S_1\circ T)$}
						\State $S \leftarrow S_2$.
					\Else
						\State $S \leftarrow S_1$.
					\EndIf
				\EndWhile
				\State $r_i \leftarrow R, s_j \leftarrow S, t_k \leftarrow T$. \Comment{Only one element exists in $R, S$ and $T$}
				\State \Return $(r_i, s_j, t_k)$.
			\end{algorithmic}
		\end{algo}
		\textbf{Time Complexity:} Denote $F(R, S, T) := (R\circ S)\circ T - R\circ(S\circ T)$. 
		Similar with the argument presented in lecture note, if it exists $r^*, s^*, t^*$ such that $F(r^*, s^*, t^*)\neq 0$, then 
		we have 
		\begin{align*}
			F(r^*, s^*, t^*) = &F(R_1, S_1, T_1) - F(R_1, S_1, T_0) - F(R_1, S_0, T_1)  - F(R_0, S_1, T_1) \\
			& F(R_1, S_0, T_0) + F(R_0, S_1, T_0) + F(R_0, S_0, T_1) - F(R_0, S_0, T_0) 
		\end{align*}
		where $R_1 = R \cup \{r^*\}, R_0 = R$, etc. The first step of \textbf{Algorithm \ref{algo:find}} terminates in $O(1)$ iterations, with a time complexity of $O(n^2)$. 

		The following steps halve the size of $R/S/T$ in each iteration, terminating in $O(\log n)$ iterations, resulting in a time complexity of $O(n^2\log n)$. 
		
		Thus, the overall time complexity of the algorithm is $O(n^2\log n)$.
	\end{enumerate}
	\ed
\end{answer}