\begin{question}{1 (15') (Primality Test with Square Root Oracle)}    
	Suppose you are given a black-box algorithm (known as ``oracle'') $\mathcal S(b,n)$ for computing square roots of $b$ modulo $n$. In other words, the algorithm may return \textit{one} $a$ such that $a^2\equiv b\pmod n$ in each invocation, or output $\bot$ if there is no root. Using this algorithm as a black box, design an $\mathsf{RP}$ algorithm (i.e., algorithm with one-side error) for compositeness, and analyze its error bound.
	
	\textit{[Hint: You do not know the behavior of algorithm $\mathcal S$. For example, the solution to $a^2\equiv 1\pmod{12}$ is $a\equiv 1,5,7,11\pmod{12}$, but $\mathcal S(1,12)$ may return $1$ all the time. You can never expect $\mathcal S$ to return a root randomly; its output can be adversarial. Therefore, in this question you must use some randomness in your algorithm.]}
\end{question}

\begin{answer}
	Consider the following algorithm (suppose $n$ is odd here):
	\begin{algo}
		\centering
		\caption{Primality Test with Square Root Oracle}
		\label{alg:p_test}
		\begin{algorithmic}[1]
			\Require A number $n$ and the oracle $\mathcal S(b,n)$.
			\Ensure Whether $n$ is prime or not.
			\State Randomly sample $b \in \mathbb{Z}_n$. If $\gcd(b,n) \neq 1$, then \Return \textbf{No} immediately.
			\State Loop to check if $n$ is a perfect power of prime $p \in [2, \log n]$.  \Comment{Time complexity: $O(\log^2 n)$}
			\State $a \leftarrow \mathcal S(b^2,n)$.
			\If{$a = \pm b$}
				\State \Return \textbf{Yes}
			\Else
				\State \Return \textbf{No}
			\EndIf
		\end{algorithmic}
	\end{algo}
	If $n$ is prime, then $a^2 \equiv b^2 \pmod{n} \implies (a-b)(a+b) \equiv 0 \pmod{n} \implies n|(a-b) \lor n|(a+b)$. 
	Therefore, $a \equiv \pm b \pmod{n}$, and the algorithm will return \textbf{Yes} with probability $1$. 
	\textbf{Algorithm \ref{alg:p_test}} is a $\mathsf{RP}$ algorithm.

	If $n$ is composite,  when $n = p^k$ for some $k$ and prime $p$, then the algorithm will return \textbf{No}. 
	Now consider $ n = pq$ where $p,q$ are distinct odd prime. And analyse the solutions of Congruence Equation 
	\begin{align}
		\label{eq:cong}
		x^2 &\equiv b^2 \pmod{pq} 
	\end{align}
	By the Chinese Remainder Theorem, we can decompose the equation into two equations:
	\begin{align*}
		x^2 &\equiv b^2 \pmod{p} \\
		x^2 &\equiv b^2 \pmod{q}
	\end{align*}
	whose solutions are:
	\begin{align*}
		x &\equiv b \pmod{p}, \quad x \equiv -b \pmod{p} \\
		x &\equiv b \pmod{q}, \quad x \equiv -b \pmod{q}
	\end{align*}
	Furthermore, we can get $4$ solutions of \eqref{eq:cong}:
	\begin{align*}
		x &\equiv \pm b \pmod{pq}, \quad x \equiv \pm (b(qq^{-1} - pp^{-1})) \pmod{pq} 
	\end{align*}
	where $q^{-1}$ and $p^{-1}$ are the inverses of $q$ and $p$ modulo $p$ and $q$, respectively.
	We can verify that the four solutions are distinct. 

	For more general cases, i.e., $n = \prod_i p_i^{k_i}$ where $p_i$ are distinct primes and $i \ge 2$. 
	Using similar argument, the congruence equation $x^2 \equiv b^2 \pmod{n}$ has at least $4$ distinct solutions.
	Therefore, 
	\[
		\Pr[\mathcal S(b,n) \not\equiv \pm b | n \text{ is composite}] \ge \frac{1}{2}. \implies \Pr[\text{Error}] \le \frac{1}{2}
	\]
	By repeating the algorithm(Line 3-7) $k$ times, we can reduce the error probability to $1/2^k$.
	\ed 
\end{answer}