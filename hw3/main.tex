\documentclass[11pt]{article}           
\usepackage[UTF8]{ctex}
\usepackage[a4paper]{geometry}
\geometry{left=2.0cm,right=2.0cm,top=2.5cm,bottom=2.25cm}

\usepackage{xcolor}
\usepackage{paralist}
\usepackage{enumitem}
\setenumerate[1]{itemsep=3pt,partopsep=0pt,parsep=0pt,topsep=0pt}
\setitemize[1]{itemsep=0pt,partopsep=0pt,parsep=0pt,topsep=0pt}
\usepackage{comment}
\usepackage{booktabs}
\usepackage{graphicx}
\usepackage{float}
\usepackage{sgame} % For Game Theory Matrices 
% \usepackage{diagbox} % Conflict with sgame
\usepackage{amsmath,amsfonts,graphicx,amssymb,bm,amsthm}
\usepackage{algorithm,algorithmicx}
% \usepackage[ruled]{algorithm2e}
% \usepackage[noend]{algpseudocode}
\usepackage{algpseudocode}

\usepackage{fancyhdr}
\usepackage{tikz}
\usepackage{pgfplots}
\pgfplotsset{compat=1.18}
\usepackage{graphicx}
\usetikzlibrary{arrows,automata}
\usepackage[hidelinks]{hyperref}
\usepackage[capitalize]{cleveref}
\usepackage{extarrows}
\usepackage{totcount}
\setlength{\headheight}{14pt}
\setlength{\parindent}{0 in}
\setlength{\parskip}{0.5 em}
\usepackage{helvet}
\usepackage{dsfont}
% \usepackage{newtxmath}
\usepackage[labelfont=bf]{caption}
\renewcommand{\figurename}{Figure}
\usepackage[english]{babel}
\usepackage{datetime}
\usepackage{lastpage}
\usepackage{istgame}
\usepackage{sgame}
\usepackage{tcolorbox}

\usepackage{tkz-graph}

\usepackage{physics}


\usepackage{etoolbox}
\AtBeginEnvironment{algorithmic}{\renewcommand{\algorithmicensure}{\textbf{Output:}}}
\AtBeginEnvironment{algorithmic}{\renewcommand{\algorithmicrequire}{\textbf{Input:}}}

\newtheorem{theorem}{Theorem}
\newtheorem{lemma}[theorem]{Lemma}
\newtheorem{proposition}[theorem]{Proposition}
\newtheorem{claim}[theorem]{Claim}
\newtheorem{corollary}[theorem]{Corollary}
\newtheorem{definition}[theorem]{Definition}
\newtheorem*{definition*}{Definition}
\newtheorem{remark}[theorem]{Remark}

\newenvironment{question}[2][Question]{\begin{trivlist}
    \item[\hskip \labelsep {\bfseries #1}\hskip \labelsep {\bfseries #2.}]\songti}{\hfill$\blacktriangleleft$\end{trivlist}}
\newenvironment{answer}[1][Answer]{\begin{trivlist}
\item[\hskip \labelsep {\bfseries #1.}\hskip \labelsep]}{\hfill$\lhd$\end{trivlist}}


\newtheorem{clm}[theorem]{Claim}

\newtheorem{fact}[theorem]{Fact}

\newcommand\1{\mathds{1}}
% \newcommand\1{\mathbf{1}}
\newcommand\R{\mathbb{R}}
\newcommand\E{\mathbb{E}}
\newcommand\N{\mathbb{N}}
\newcommand\NN{\mathcal{N}}
\newcommand\per{\mathrm{per}}
% \newcommand\Pr{\mathbb{Pr}}
\newcommand\Var{\mathrm{Var}}
\newcommand\Cov{\mathrm{Cov}}
\newcommand{\Exp}{\mathrm{Exp}}
\newcommand{\arrp}{\xrightarrow{P}}
\newcommand{\arrd}{\xrightarrow{d}}
\newcommand{\arras}{\xrightarrow{a.s.}}
\newcommand{\arri}{\xrightarrow{n\rightarrow\infty}}
\newcommand{\ed}{\textbf{Q.E.D.}}

\definecolor{lightgray}{gray}{0.75}

\makeatletter
\newenvironment{algo}
  {% \begin{breakablealgorithm}
    \begin{center}
      \refstepcounter{algorithm}% New algorithm
      \hrule height.8pt depth0pt \kern2pt% \@fs@pre for \@fs@ruled
      \parskip 0pt
      \renewcommand{\caption}[2][\relax]{% Make a new \caption
        {\raggedright\textbf{\fname@algorithm~\thealgorithm} ##2\par}%
        \ifx\relax##1\relax % #1 is \relax
          \addcontentsline{loa}{algorithm}{\protect\numberline{\thealgorithm}##2}%
        \else % #1 is not \relax
          \addcontentsline{loa}{algorithm}{\protect\numberline{\thealgorithm}##1}%
        \fi
        \kern2pt\hrule\kern2pt
     }
  }
  {% \end{breakablealgorithm}
     \kern2pt\hrule\relax% \@fs@post for \@fs@ruled
   \end{center}
  }
\makeatother

\DeclareMathOperator*{\argmax}{argmax} % 定义 \argmax 运算符
\DeclareMathOperator*{\argmin}{argmin} % 定义 \argmin 运算符

\newcommand{\ta}[1]{{\color{red!100!teal} [{\bf TA's Note:} #1]}}


\title{Homework Set \#1}
\usetikzlibrary{positioning}

\begin{document}

    \pagestyle{fancy}
    \lhead{Peking University}
    \chead{\kaishu }
    \rhead{Randomized Algorithm, Spring 2025}
    \fancyfoot[R]{} 
    \fancyfoot[C]{\thepage\ /\ \pageref{LastPage} \\ \textcolor{lightgray}{Last Compile: \today}}

    \begin{center}
        {\LARGE \bf Homework \#3}\\
        {Due: 2025-5-8 23:59 \quad$|$\quad 6 Questions, 100 Pts}\\
        {Name: Jiacong Fang, ID: 2200017849}
    \end{center}

    \textbf{Note: }The total points of this homework is 10 + 15 + 5 + 30 + 10 + 30 = 100.

    \begin{question}{1 (15') (Primality Test with Square Root Oracle)}    
	Suppose you are given a black-box algorithm (known as ``oracle'') $\mathcal S(b,n)$ for computing square roots of $b$ modulo $n$. In other words, the algorithm may return \textit{one} $a$ such that $a^2\equiv b\pmod n$ in each invocation, or output $\bot$ if there is no root. Using this algorithm as a black box, design an $\mathsf{RP}$ algorithm (i.e., algorithm with one-side error) for compositeness, and analyze its error bound.
	
	\textit{[Hint: You do not know the behavior of algorithm $\mathcal S$. For example, the solution to $a^2\equiv 1\pmod{12}$ is $a\equiv 1,5,7,11\pmod{12}$, but $\mathcal S(1,12)$ may return $1$ all the time. You can never expect $\mathcal S$ to return a root randomly; its output can be adversarial. Therefore, in this question you must use some randomness in your algorithm.]}
\end{question}

\begin{answer}
	Consider the following algorithm (suppose $n$ is odd here):
	\begin{algo}
		\centering
		\caption{Primality Test with Square Root Oracle}
		\label{alg:p_test}
		\begin{algorithmic}[1]
			\Require A number $n$ and the oracle $\mathcal S(b,n)$.
			\Ensure Whether $n$ is prime or not.
			\State Randomly sample $b \in \mathbb{Z}_n$. If $\gcd(b,n) \neq 1$, then \Return \textbf{No} immediately.
			\State Loop to check if $n$ is a perfect power of prime $p \in [2, \log n]$.  \Comment{Time complexity: $O(\log^2 n)$}
			\State $a \leftarrow \mathcal S(b^2,n)$.
			\If{$a = \pm b$}
				\State \Return \textbf{Yes}
			\Else
				\State \Return \textbf{No}
			\EndIf
		\end{algorithmic}
	\end{algo}
	If $n$ is prime, then $a^2 \equiv b^2 \pmod{n} \implies (a-b)(a+b) \equiv 0 \pmod{n} \implies n|(a-b) \lor n|(a+b)$. 
	Therefore, $a \equiv \pm b \pmod{n}$, and the algorithm will return \textbf{Yes} with probability $1$. 
	\textbf{Algorithm \ref{alg:p_test}} is a $\mathsf{RP}$ algorithm.

	If $n$ is composite,  when $n = p^k$ for some $k$ and prime $p$, then the algorithm will return \textbf{No}. 
	Now consider $ n = pq$ where $p,q$ are distinct odd prime. And analyse the solutions of Congruence Equation 
	\begin{align}
		\label{eq:cong}
		x^2 &\equiv b^2 \pmod{pq} 
	\end{align}
	By the Chinese Remainder Theorem, we can decompose the equation into two equations:
	\begin{align*}
		x^2 &\equiv b^2 \pmod{p} \\
		x^2 &\equiv b^2 \pmod{q}
	\end{align*}
	whose solutions are:
	\begin{align*}
		x &\equiv b \pmod{p}, \quad x \equiv -b \pmod{p} \\
		x &\equiv b \pmod{q}, \quad x \equiv -b \pmod{q}
	\end{align*}
	Furthermore, we can get $4$ solutions of \eqref{eq:cong}:
	\begin{align*}
		x &\equiv \pm b \pmod{pq}, \quad x \equiv \pm (b(qq^{-1} - pp^{-1})) \pmod{pq} 
	\end{align*}
	where $q^{-1}$ and $p^{-1}$ are the inverses of $q$ and $p$ modulo $p$ and $q$, respectively.
	We can verify that the four solutions are distinct. 

	For more general cases, i.e., $n = \prod_i p_i^{k_i}$ where $p_i$ are distinct primes and $i \ge 2$. 
	Using similar argument, the congruence equation $x^2 \equiv b^2 \pmod{n}$ has at least $4$ distinct solutions.
	Therefore, 
	\[
		\Pr[\mathcal S(b,n) \not\equiv \pm b | n \text{ is composite}] \ge \frac{1}{2}. \implies \Pr[\text{Error}] \le \frac{1}{2}
	\]
	By repeating the algorithm(Line 3-7) $k$ times, we can reduce the error probability to $1/2^k$.
	\ed 
\end{answer}
    \clearpage
    
    \begin{question}{2 (15') (Distinguish Sets by Intersections)}
    Consider $k \leq \frac{1.99n}{\log_2{n}}$ and assume $n$ is sufficiently large, prove that for any collection of subsets $S_1, \cdots, S_k \subseteq \qty{1, \cdots, n}$, there exist two distinct subsets $X, Y \subseteq \qty{1, \cdots, n}$ such that $\abs{X \cap S_i} = \abs{Y \cap S_i}$ for all $i \in \qty{1, \cdots, k}$.

    \textit{[Hint: Introduce proper randomization process when choosing $X$, and analyze the concentration of $\abs{X\cap S_i}$.]}
    \end{question}


\begin{answer}
    Given any collection of subsets $S_1, \cdots, S_k \subseteq \qty{1, \cdots, n}$, let $k_i = |S_i|$. 
    We randomly choose $X \subseteq \qty{1, \cdots, n}$ by including each element with probability $p = 1/2$. 
    Given $X$ and for any $i\in \{1,2, \cdots, k\}$, denote $N_i = |X \cap S_i|$. 
    Then $N_i$ follows a binomial distribution, $N_i \sim \text{Binomial}(k_i, 1/2)$.
    Therefore, by the Chernoff bound, we have
    \begin{equation*}
        \Pr\left[\left|N_i - \frac{k_i}{2}\right| \ge \varepsilon\right] \leq 2\exp\left(-\frac{2 \varepsilon^2}{k_i}\right)
    \end{equation*}
    where $\varepsilon$ is a constant waited to be determined. Then by the union bound, we have
    \begin{align*}
        \Pr\left[\exists i \text{ s.t. }  \left|N_i - \frac{k_i}{2}\right| \ge \varepsilon \right] &\le \sum_{i=1}^k \Pr\left[\left|N_i - \frac{k_i}{2}\right| \ge \varepsilon\right]  \le 2 \sum_{i=1}^k \exp\left(-\frac{2 \varepsilon^2}{k_i}\right) \\
        &\le 2k \cdot \exp\left(-\frac{2 \varepsilon^2}{n}\right) := p(\varepsilon)
    \end{align*}
    We choose the $X$ that satisfies $\left|N_i - k_i / 2\right| < \varepsilon$ for all $i\in \{1,2, \cdots, k\}$, 
    then the number of such $X$ is at least $N_\text{valid} := 2^n \cdot (1 - p(\varepsilon))$. 
    Notice that the number of valid $v = (N_1, N_2, \cdots, N_k)$ is at most $N_\text{vec} := (2\varepsilon)^k$. 
    Then our goal is to show that $N_\text{valid} >  N_\text{vec}$, which implies the existence of two distinct satisfied $X$ and $Y$. 

    Let $\varepsilon = \sqrt{n\ln (2k^2)} / \sqrt{2}$. Then we have
    \begin{align*}
        p(\varepsilon) \le 2k \cdot e^{-\ln (2k^2)}  = \frac{1}{k}. \quad N_\text{vec} = \left(2n\ln (2k^2)\right)^{k/2}
    \end{align*}
    Therefore, we only need to show that
    \begin{align*}
        2^n \cdot \frac{k-1}{k} < \left(2n\ln (2k^2)\right)^{k/2} &\iff \frac{k}{2} \log_2 \left[2n \ln (2k^2)\right] < n + \log_2 \frac{k-1}{k}  \\
        &\iff k < \frac{2n + 2\log_2 [(k-1)/k]}{\log_2 \left[2n \ln (2k^2)\right]} 
    \end{align*}
    Notice that $ k \leq \frac{1.99n}{\log_2{n}}$, so we need to show that
    \begin{equation}
        \frac{1.99n}{\log_2{n}} \le  \frac{2n + 2\log_2 [(k-1)/k]}{\log_2 \left[2n \ln (2k^2)\right]} = \frac{2n + 2\log_2 [(k-1)/k]}{\log_2 n + 1 + \log_2 [\ln (2k^2)]} \\
    \end{equation}
    which is equivalent to
    \begin{equation}
        \label{eq:2}
        0.01 n\log_2 n \ge 1.99n + 1.99n \cdot \log_2 \left[\ln (2k^2)\right] + 2\log_2 n \cdot \log_2 \left[\frac{k}{k-1}\right]
    \end{equation}
    Since we have $2\log_2 n \cdot \log_2 \left[\frac{k}{k-1}\right] = \order{\log n \cdot \log k } = \order{\log^2 n}$,  $0.01 n\log_2 n = \order{n \log n}$, and 
    \begin{equation*}
        1.99n \cdot \log_2 \left[\ln (2k^2)\right] = \order{n \log(\log k)}  = \order{n \log(\log n)} 
    \end{equation*}
    Therefore, for sufficiently large $n$, \cref{eq:2} holds, i.e., $N_\text{valid} > N_\text{vec}$, which implies the existence of two distinct $X$ and $Y$ that satisfy the requirements.
    \ed
\end{answer}
    \clearpage
    \begin{question}{3 (20') (Schwartz--Zippel is More General)} 
    Our course begins with two examples of randomized algorithms: checking matrix multiplication and checking associativity. It turns out that both can be tackled using Schwartz--Zippel's algorithm. In this problem, you will give equal or better error bounds by constructing polynomials for the two problems.
 
	\begin{itemize}
		\item[a. (10')] Given a binary operator $\circ$ on a set $X$ of size $n$, we wish to decide if $\circ$ is associative. In our class, we defined operator $\circ$ on $2^{X}$ to check the associativity on $X=\{x_1,\cdots,x_n\}$. Similarly we can define operator $\circ$ on $p^X$ where $p\geq 5$ is a given prime. In other words, every element in $p^X$ are in the form of $\sum_{i=1}^{n}r_ix_i$ with $r_i\in\mathbb{F}_p$. 
		
		Base on this, please design a randomized algorithm similar to the one in class. However, this time, you may want to analyze its error bound by applying Schwartz--Zippel Lemma to a polynomial, rather than inclusive-exclusive argument. (You do not need to prove the equivalence to associativity. Schwartz--Zippel Lemma also works for finite fields, provided the size of field is larger than the degree of polynomial --- you do not need to prove this, either. )
		
		\textit{[Hint: Construct multivariate polynomials $F_1(\cdot),\cdots,F_n(\cdot)$ such that $\sum_{i=1}^{n}F_i(\cdot)x_i\equiv 0$ iff $\circ$ is associative over $p^X$. ]}
		
		\item [b. (10')] Design a randomized algorithm that finds a counterexample when $\circ$ is not associative over $X$. Analyze the time complexity of your algorithm.
	\end{itemize}
\end{question} 

\begin{answer}
	\begin{enumerate}[label=\alph*).]
		\item The algorithm is as follows:
		\begin{algo}
			\caption{\textbf{Check Associativity}}
			\label{algo:check}
			\begin{algorithmic}[1]
				\Require A binary operator $\circ$ on a set $X$ of size $n$.
				\Ensure Whether $\circ$ is associative over $p^X$
				\State Randomly sample $\{r_i\}_{i=1}^n, \{s_j\}_{j=1}^n, \{t_k\}_{k=1}^n$ from $\mathbb{F}_p$ independently and uniformly.
				\State $R \leftarrow \sum_{i=1}^{n}r_ix_i, S \leftarrow \sum_{j=1}^{n}s_jx_j, T \leftarrow \sum_{k=1}^{n}t_kx_k$.
				\If{$(R\circ S)\circ T\neq R\circ(S\circ T)$}
					\State \Return \text{False}
				\Else 
					\State \Return \text{True}
				\EndIf
			\end{algorithmic}
		\end{algo}
		Let $F(x) = (R\circ S)\circ T - R\circ(S\circ T)$, and denote $F_i(\cdot)$ as the coefficient of $x_i$ in $F(x)$. Then we have
			$F(x) = \sum_{i=1}^{n}F_i(\cdot)x_i. $
		And it's easy to verify that $F(x) \equiv 0$ iff $\circ$ is associative over $p^X$. Note that $\deg(F(x)) \le 3$. 
		By Schwartz-Zippel Lemma, the error bound of this algorithm is 
		\begin{align*}
			\Pr[R\circ S\circ T = R\circ S\circ T|\text{$\circ$ is not associative}] \le \frac{3}{|\mathbb{F}_p|} = \frac{3}{p}.
		\end{align*}
		\item We first repeat \textbf{Algorithm 1} until finding $R, S, T$ such that $(R\circ S)\circ T\neq R\circ(S\circ T)$. 
		And then find the counterexample $r_i, s_j, t_k \in X$ by the following algorithm:
		\begin{algo}
			\caption{\textbf{Find Counterexample}}
			\label{algo:find}
			\begin{algorithmic}[1]
				\Require A binary operator $\circ$ on a set $X$ of size $n$.
				\Ensure A counterexample $(r_i, s_j, t_k)$ such that $(r_i\circ s_j)\circ t_k\neq r_i\circ(s_j\circ t_k)$.
				\State Repeat \textbf{Algorithm \ref{algo:check}} until finding $R, S, T$ such that $(R\circ S)\circ T\neq R\circ(S\circ T)$.
				\State Denote $R = \sum_{i=1}^{n}r_ix_i, S = \sum_{j=1}^{n}s_jx_j, T = \sum_{k=1}^{n}t_kx_k$.
				\While{$\max\{|R|, |S|, |T|\} > 1$} \Comment{$|R| := \#\{r_i|r_i \neq 0\}$, etc.}
					\State Split $S = \sum_{i=1}^{\lfloor n/2\rfloor}s_ix_i + \sum_{i=\lfloor n/2\rfloor+1}^{n}s_ix_i := S_1 + S_2$. \Comment {assume $|S| = \max\{|R|, |S|, |T|\}$}
					\If{$(R\circ S_1)\circ T = R\circ(S_1\circ T)$}
						\State $S \leftarrow S_2$.
					\Else
						\State $S \leftarrow S_1$.
					\EndIf
				\EndWhile
				\State $r_i \leftarrow R, s_j \leftarrow S, t_k \leftarrow T$. \Comment{Only one element exists in $R, S$ and $T$}
				\State \Return $(r_i, s_j, t_k)$.
			\end{algorithmic}
		\end{algo}
		\textbf{Time Complexity:} Denote $F(R, S, T) := (R\circ S)\circ T - R\circ(S\circ T)$. 
		Similar with the argument presented in lecture note, if it exists $r^*, s^*, t^*$ such that $F(r^*, s^*, t^*)\neq 0$, then 
		we have 
		\begin{align*}
			F(r^*, s^*, t^*) = &F(R_1, S_1, T_1) - F(R_1, S_1, T_0) - F(R_1, S_0, T_1)  - F(R_0, S_1, T_1) \\
			& F(R_1, S_0, T_0) + F(R_0, S_1, T_0) + F(R_0, S_0, T_1) - F(R_0, S_0, T_0) 
		\end{align*}
		where $R_1 = R \cup \{r^*\}, R_0 = R$, etc. The first step of \textbf{Algorithm \ref{algo:find}} terminates in $O(1)$ iterations, with a time complexity of $O(n^2)$. 

		The following steps halve the size of $R/S/T$ in each iteration, terminating in $O(\log n)$ iterations, resulting in a time complexity of $O(n^2\log n)$. 
		
		Thus, the overall time complexity of the algorithm is $O(n^2\log n)$.
	\end{enumerate}
	\ed
\end{answer}
    \clearpage
    \begin{question}{4 (30') (Random Graphs Proof of Power of Two Choices)} In this problem, we will show another way to prove that the maximium load is $O(\log \log n)$, though our argument cannot provide the coefficient. 
    
    There are $\frac{n}{512}$ balls and $n$ bins. We implement the two-choice procedure and record the procedure using a graph with $n$ nodes:
    
% \begin{algorithm}[H]
% \SetAlgoLined
% \KwResult{load,$G$}
% $G=(V,\emptyset)$\;
% $\forall i, \mathrm{load}(i)=0$\;
% \For{each ball}{
% Randomly pick two bins $u\neq v$\;
% Add edge $(u,v)$ to $G$\;
% \eIf{$\mathrm{load}(u)\leq \mathrm{load}(v)$}{$\mathrm{load}(u)++$\tcc*{throw the ball into bin $u$}}{ $\mathrm{load}(v)++$\tcc*{throw the ball into bin $v$}}
% }
% \caption{Power of two choice}
% \end{algorithm}

\begin{algo}
    \caption{Power of Two Choices}
    \label{alg:two-choices}
    \centering
    \begin{algorithmic}[1]
        \Ensure{Load, $G$}
        \State $G \leftarrow (V,\emptyset)$
        \State $\forall i, \mathrm{load}(i) \leftarrow 0$
        \For{each ball}
            \State Randomly pick two bins $u\neq v$
            \State Add edge $(u,v)$ to $G$
            \If{$\mathrm{load}(u)\leq \mathrm{load}(v)$}
                \State $\mathrm{load}(u) \leftarrow \mathrm{load}(u) + 1$ \Comment{\texttt{throw the ball into bin $u$}}
            \Else
                \State $\mathrm{load}(v) \leftarrow \mathrm{load}(v) + 1$ \Comment{\texttt{throw the ball into bin $v$}}
            \EndIf
        \EndFor
    \end{algorithmic}
\end{algo}

We use the graph $G$ to analyze the loads. 
    \begin{itemize}
        \item [a. (10')] Show that there exists a constant $K > 0$ such that, for all subsets $S$ of the vertex set with $|S| \ge K$, the induced graph $G[S]$ contains at most $5|S|/2$ edges, and hence has average degree at most $5$, w.h.p.
        
        \textit{[Hint: You may find this useful: $\binom{n}{d}\le \left(\frac{ne}{d}\right)^d$.]}
        
        \textit{[Hint: When you attempt to bound a sum, try to break it into two parts and bound them separately.]}
        
        \item [b. (10')] Given graph $G$, we recursively remove all vertices of degree $\le 10$ in $G$ until there are no more vertices of degree $\le 10$. Prove that this procedure ends after $O(\log \log n)$ rounds w.h.p., and the number of remaining vertices in each remaining component is at most $K$ w.h.p.
        
        \item [c. (10')] $\forall i$, if a node $u$ survives $i$ rounds, show that $\mathrm{load}(u)\leq 10 i$. If a node $u$ is never deleted, show that $\mathrm{load}(u)\leq 10i^* + cK$ w.h.p., where $i^*$ is the total number of rounds and $c$ is a constant.
        
        \textit{[Hint: Use Induction.]}
    \end{itemize}

    (b) shows w.h.p. that $i^{*} = O(\log \log n)$. Thus, if we throw $n/512$ balls into $n$ bins using the best-of-two-bins method, then w.h.p. the maximum load of any bin is $O(\log \log n)$. Hence for the case of $n$ balls and $n$ bins, the maximum load would be at most $512*O(\log \log n)=O(\log \log n)$.
    
\end{question}

\begin{answer}
    \begin{enumerate}[label=\alph*).]
        \item For a fixed set $S_0$ with $k$ nodes, we have (denote $m = n / 512$):  
        \begin{equation*}
            \Pr\left[S_0 \text{ has } > \frac{5k}{2} \text{ edges}\right] \le \binom{m}{5k/2} \left(\frac{k}{n}\right)^{2.5k \times 2} = \binom{m}{5k/2} \left(\frac{k}{n}\right)^{5k}
        \end{equation*}
        since each edge has $2$ endpoints, and the probability of each vertex being selected into $S$ is $k / n$ while the choices are independent. Then by the union bound, we have:
        \begin{align*}
            \Pr\left[\exists \text{$S$ that contains } > \frac{5|S|}{2} \text{ edges}\right] &\le \sum_{k \ge K} \binom{n}{k} \binom{m}{5k/2} \left(\frac{k}{n}\right)^{5k}  \\
            \text{(by hint 1) }&\le \sum_{k\le K} \left(\frac{ne}{k}\right)^k \left(\frac{ne}{512(5k/2)}\right)^{5k/2} \left(\frac{k}{n}\right)^{5k} \\
            &= \sum_{k \le K} \left(\frac{k}{n}\right)^{3k/2} \frac{e^{7k/2}}{1280^{5k/2}} \\
            &= \sum_{k \le K} \left(\frac{k}{n}\right)^{3k/2} \left(\frac{e^{3.5}}{1280^{2.5}}\right)^k.
        \end{align*}
        Let $ a = \frac{e^{3.5}}{1280^{2.5}} < 1/2$, then for large enough $k$, $a^k$ will be exponentially small, and for small $k$, we can bound the sum by the other terms, i.e. $(k/n)^{3k/2}$.  Therefore, we bound the sum by dividing it into two parts:
        \begin{align*}
            \Pr\left[\exists \text{$S$ that contains } > \frac{5|S|}{2} \text{ edges}\right] &\le \sum_{k = K}^{2\log_2 n} \left(\frac{k}{n}\right)^{3k/2} a^k + \sum_{k = 2\log_2 n}^{n} \left(\frac{k}{n}\right)^{3k/2} a^k \\
            &\le \sum_{k = K}^{2\log_2 n} \left(\frac{k}{n}\right)^{3k/2} + \sum_{k = 2\log_2 n}^{n} a^k \\
            &\le 2\log_2 n \cdot \left(\frac{2\log_2 n}{n}\right)^{3K/2} + \sum_{k = 2\log_2 n}^{n} \frac{1}{n^2} \\
            &\le \order{\frac{1}{n}}
        \end{align*}
        The last inequality holds when we choose $K$ to be a large enough constant. Therefore, we have shown that the statement holds with high probability.
        \item We first prove that the size of $G$'s largest connected component is $O(\log n)$ with high probability. Consider $k = \Theta(\log n)$, then we have:
        \begin{align*}
            \Pr\left[k+1 \text{ vertices are connected}\right] &\le \Pr\left[\text{at least $k$ edges within the $k+1$ vertices}\right] \\
            &\le \binom{m}{k} \left(\frac{\binom{k+1}{2}}{\binom{n}{2}}\right)^{k} = \binom{m}{k} \left(\frac{k(k+1)}{n(n-1)}\right)^k \\
            (n(n-1)\ge n^2/2, k(k+1) \le 2k^2)~&\le \binom{m}{k} \left(\frac{2k^2}{n^2/2}\right)^k = \binom{m}{k} \left(\frac{4k^2}{n^2}\right)^k
        \end{align*}
        Therefore, by the union bound, we have:
        \begin{align*}
            \Pr\left[\exists\text{ a connected component of size $k+1$}\right] &\le \binom{n}{k+1} \Pr\left[k+1 \text{ vertices are connected}\right] \\
            &\le \binom{n}{k+1} \binom{m}{k} \left(\frac{4k^2}{n^2}\right)^k \\
           \left(\binom{n}{d}\le \left(\frac{ne}{d}\right)^d\right)\quad &\le \left(\frac{ne}{k+1}\right)^{k+1} \left(\frac{ne}{512k}\right)^k \left(\frac{4k^2}{n^2}\right)^k 
        \end{align*}
        \begin{align*}
            \Pr\left[\exists\text{ a connected component of size $k+1$}\right] & \le \binom{m}{k} \left(\frac{2k^2}{n^2/2}\right)^k = \binom{m}{k} \left(\frac{4k^2}{n^2}\right)^k \\
            &= \frac{en}{k+1} \left(\frac{k}{k+1}\right)^k \left(\frac{e^2}{128}\right)^k   \\
            (k = \Theta(\log n))\quad &\le \frac{en}{k+1} \left(\frac{e^2}{128}\right)^k \le \order{\frac{1}{n}}.
        \end{align*}
        Therefore, we have shown that the size of $G$'s largest connected component is $O(\log n)$ with high probability.
        
        When there exists a component with more than $K$ vertices, by the result of part (a), we know that the average degree of the component is at most $5$. 
        Then by Markov's inequality, for any vertex $u$ in the component, we have:
        \begin{align*}
            \Pr\left[\deg(u) \ge 11\right] &\le \frac{\E_u\left[\deg(u)\right]}{11} \le \frac{5}{11} < \frac{1}{2}.
        \end{align*}
        Therefore, in expectation, we have at least $K/2$ vertices with degree at most $10$ and will be deleted in current round. 

        Combining these two results, we observe that in each round, the number of vertices in any component with size larger than $K$ is reduced by at least half in expectation. Since the size of the largest component is initially at most $O(\log n)$, the number of rounds is at most $O(\log \log n)$ with high probability. And when the procedure ends, the number of remaining vertices in each component is at most $K$ with high probability.
        \item We prove it by induction on $i$. For the base case, if a node $u$ is deleted in the $i=1$ round, then $\deg(u) \le 10$,  so the load of $u$ is at most $10$. The statement holds for $i=1$.
        
        Suppose the statement holds for $i \in \{1,2,\cdots, k-1\}(k\le i^*)$, now consider any vertex deleted in the $k$-th round. Denote the set of vertices that are deleted in the first $k-1$ rounds as $S_{k-1}$, then by 
        the induction hypothesis, we have:
        \begin{equation*}
            \forall s \in S_{k-1},\quad  \text{load}(s) \le 10(k-1).
        \end{equation*}
        Since each ball is thrown into the bin with the smaller load, the load of $u$ that relates to $S_{k-1}$ is at most $10(k-1)$. 
        And $u$ is deleted in the $k$-th round which means that $\deg(u) \le 10$, so the total laod of $u$ is at most $10k$. Therefore, the statement holds for $i=k$.

        Now we consider the case when $u$ is never deleted, denote the set of vertices that are deleted in the first $i^*$ rounds as $S_{i^*}$. Similar to the previous case, we know that 
        the load of $u$ that relates to $S_{i^*}$ is at most $10i^*$. Besides, due to the result of part (b), we know that the size of the largest component is at most $K$ with high probability.
        Therefore, the load of $u$ is at most $10i^* + cK$ with high probability.

        Combining these two results, we have shown that $\forall i$, if a node $u$ survives $i$ rounds, then $\mathrm{load}(u)\leq 10 i$. If a node $u$ is never deleted, then $\mathrm{load}(u)\leq 10i^* + cK$ with high probability.
    \end{enumerate}
    \ed
\end{answer}
    \clearpage 
    
\begin{question}{5 (20') (Checking Commutativity)}
	Let $G$ be a finite group. In this question we want to check if $G$ is abelian, i.e., whether
	$$
	x\circ y=y\circ x
	$$
	holds for all $x,y\in G$. We assume $G$ is presented as a set of \textit{generators} $\{g_1,g_2,\cdots,g_k\}$ (i.e., every element of $G$ can be written as a product of various $g_i$) together with a black box which returns any desired product $g_i\circ g_j$ in unit time.
	
	There is a randomized algorithm making use of a \textit{random product} $h=g_1^{b_1}\circ g_2^{b_2}\circ\cdots\circ g_k^{b_k}$, where $b_i\in\{0,1\}$ are independent fair coin flips. It runs as follows
	\begin{enumerate}
		\item Let $h,h'$ be two independent \textit{random products}.
		\item If $h\circ h'\neq h'\circ h$, then output ``not abelian''; otherwise output ``abelian''.
	\end{enumerate}
	
	Plainly this algorithm runs in $O(k)$ time, and is always correct when $G$ is abelian. Here are the questions.
	\begin{itemize}
		\item[a. (10')] Suppose $H$ is any \textit{proper} subgroup of $G$. For a random product $h$, show that 
		$$
		\Pr[h\notin H]\geq\frac{1}{2}.
		$$
		\textit{[Hint: Construct $h'\notin H$ from any $h\in H$. ]}
		
		\item[b. (10')] Show that when $G$ is not abelian, the algorithm reports a correct answer with probability at least $1/4$. 
		
		\textit{[Hint: Consider the elements of $G$ that commute with all elements of $G$.]}
	\end{itemize}
\end{question}

\begin{answer}
	\begin{enumerate}[label=\alph*).]
		\item $H < G$ implies that there exists $a \in G$ such that $a \notin H$. Therefore, for any random $h \in H$, $h \circ a \notin H$. And for any $h_1, h_2 \in H, h_1\neq h_2$, it's easy to verify that $h_1\circ a \neq h_2\circ a$. 
		Then for any random product $h$, we have 
		\begin{align*}
			1 = \Pr[h\in H] + \Pr[h\notin H] \le 2\Pr[h\notin H] \implies \Pr[h\notin H] \ge \frac{1}{2}.
		\end{align*}
		\item 
	\end{enumerate}
\end{answer}
    \clearpage
    \begin{question}{6 (20') (New Pattern Matching)}
	In this question we use a different fingerprinting technique for pattern matching. The idea is to map bit string $s$ into a $2\times2$ matrix $M(s)$ as follows:
	\begin{itemize}
		\item For empty string $\varepsilon$, $M(\varepsilon)=\begin{bmatrix}1&0\\0&1\end{bmatrix}$.
		\item $M(0)=\begin{bmatrix}1&0\\1&1\end{bmatrix},M(1)=\begin{bmatrix}1&1\\0&1\end{bmatrix}.$
		\item For non-empty strings $x,y$, $M(xy)=M(x)M(y).$
	\end{itemize}
	
	Here are the questions. 
	\begin{itemize}
		\item[a. (10')] Show that this fingerprint function has the following properties.
		\begin{enumerate}
			\item (2') $M(x)$ is well-defined for all $x\in\{0,1\}^*$.
			\item (8') $M(x)=M(y)\implies x=y$.
		\end{enumerate}
		\item[b. (10')] By considering the matrices modulo a suitable prime $p$, show an efficient randomized algorithm for pattern matching. Analyze the error bound of your algorithm.
	\end{itemize}
\end{question}

\begin{answer}
	\begin{enumerate}[label=\alph*).]
		\item \underline{For \textbf{property 1}}, let $x := x_1x_2\cdots x_n \in \{0,1\}^*$, then indection on $n$ shows that $M(x)$ is well-defined, i.e., 
		\begin{align*}
			M(x) = M(\varepsilon) \prod_{i=1}^{n} M(x_i).
		\end{align*}
		\begin{itemize}
			\item \textbf{Base case:} When $n = 1$, it's easy to verify that 
			\begin{align*}
				M(\varepsilon)M(1)= \begin{bmatrix}1&0\\0&1\end{bmatrix}\cdot \begin{bmatrix}1&1\\0&1\end{bmatrix} = M(1). \quad \text{Similarly for $M(0)$.} 
			\end{align*}
			\item \textbf{Inductive Hypothesis:} Assume that the statement holds for all $n\le k$.
			\item \textbf{Inductive Step:} When $n = k+1$, we have 
			\[
				M(x_1x_2\cdots x_{k+1}) = M(x_1x_2\cdots x_k)M(x_{k+1}) = M(\varepsilon) \prod_{i=1}^{k} M(x_i)\cdot M(x_{k+1}).
			\]
			And for any $rs \in \{0,1\}^{k+1}$, if$r = \varepsilon$ or $s = \varepsilon$, then $M(rs) = M(r)M(s)$ holds by definition. Otherwise, by induction hypothesis, we have $M(rs) = M(\varepsilon) \prod_{i=1}^{k_r} M(r_i) \prod_{j=1}^{k_s} M(s_j) = M(r)M(s)$.
		\end{itemize}
		This completes the proof of \textbf{property 1}.
		
		\underline{For \textbf{property 2}}, we fisrt prove that for all $x\in \{0,1\}^*$, the elements of $M(x)$ are all non-negative. Note that $M(\varepsilon)$ and $M(0), M(1)$ are all non-negative. Then for all $x = x_1x_2\cdots x_n$, we can easily verify it by induction on $n$.

		Then we prove that $M(x) = M(y) \implies x = y$ by induction on $|x|:=n$ and $|y|:=m$.
		\begin{itemize}
			\item \textbf{Base case:} When $n = 0$, for all $m \in \N^*$, denote $y = y_1y_2\cdots y_m$.  If $y_1 = 0$, then 
			\[
				M(\varepsilon) = M(0)M(y_2\cdots y_m) \implies M(y_2\cdots y_m) = M(y) = \begin{bmatrix} 1 & 0\\ -1 & 1 \end{bmatrix} 
			\]
			which is impossible. Similarly for $y_1 = 1$. Thus, $M(\varepsilon) = M(y) \implies x = y, \forall m\in\N^*$ holds. 
			Note that the case $m = 0$ is similar, i.e., $M(x) = M(\varepsilon) \implies x = y, \forall n\in\N^*$.
			\item \textbf{Inductive Hypothesis:} Assume that the statement holds for all $n\le k_1$ and $m\le k_2$.
			\item \textbf{Inductive Step:} When $n = k_1+1$ and $m = k_2+1$. Let $x = x_1x_2\cdots x_{k_1+1}$ and $y = y_1y_2\cdots y_{k_2+1}$. If $x_{k_1+1} = y_{k_2+1}$, 
			then 
			\begin{gather*}
				M(x_1x_2\cdots x_{k_1})M(x_{k_1+1}) = M(y_1y_2\cdots y_{k_2})M(y_{k_2+1}) \\
				M(x_1x_2\cdots x_{k_1}) = M(y_1y_2\cdots y_{k_2}) \implies x_1x_2\cdots x_{k_1} = y_1y_2\cdots y_{k_2} \implies x = y.
			\end{gather*}
			Otherwise, without loss of generality, suppose $x_{k_1+1} = 0$ and $y_{k_2+1} = 1$. Let 
			\begin{align*}
				M(x_1x_2\cdots x_{k_1}) = \begin{bmatrix} a & b\\ c & d \end{bmatrix}, \quad M(y_1y_2\cdots y_{k_2}) = \begin{bmatrix} a' & b'\\ c' & d' \end{bmatrix}.
			\end{align*}
			where $a,b,c,d,a',b',c',d'$ are all non-negative. Then we have
			\begin{align*}
				M(x) = \begin{bmatrix}
					a+b & b \\
					c+d & d
				\end{bmatrix} = \begin{bmatrix}
					a' & a'+b' \\
					c' & c'+d'
				\end{bmatrix} = M(y) \implies a = b' = c = d' = 0.
			\end{align*}
			Then $\det(M(x_1x_2\cdots x_{k_1})) = 0$, which is impossible.  
		\end{itemize}
		This completes the proof of \textbf{property 2}.
		\item Like the fingerprinting technique in the lecture notes, the algorithm is as follows:
		
		\textbf{All operations are modulo $p$, ignore in the following algorithm for simplicity.}
		\begin{algo}
			\caption{\textbf{Pattern Matching}}
			\begin{algorithmic}[1]
				\Require A text $Q$ of length $n$ and a pattern $P$ of length $m$.
				\Ensure Whether $P$ appears in $Q$.
				\State Pick a random prime $p\in [2, \cdots, T]$.
				\State Compute $M(P)$ and $M(Q[0:m])$. 
				\For{$i = 1$ to $n-m$}
					\State $M(Q[i:i+m]) \leftarrow M^{-1}(Q[i-1])\cdot M(Q[i-1:i+m-1])\cdot M(Q[i+m-1])$.
					\If{$M(Q[i+1:i+m]) = M(P)$}
						\State \Return \textbf{True}.
					\Else
						\State Continue.
					\EndIf
				\EndFor
				\State \Return \textbf{False}.
			\end{algorithmic}
		\end{algo}
		\textbf{Error Bound:} the length of each element in the matrix is at most $m$. 
		In order for a false match to occur, $p$ must divide $(M(Q[i:i+m]) - M(P))$ for some $i$. 
		Thus, $p$ must divide $\prod_{i} M(Q[i:i+m]) - M(P)$. 
		Therefore, the error bound is
		\[
			\Pr[\text{Error}] \le \frac{\pi(mn)}{\pi(T)}.
		\]
	\end{enumerate}
	\ed
\end{answer}

    \clearpage
    
\end{document}