\documentclass[11pt]{article}

% ========== 页面布局与格式 ==========
\usepackage[a4paper, left=2.0cm, right=2.0cm, top=2.5cm, bottom=2.5cm]{geometry}
\usepackage{fancyhdr}
\setlength{\headheight}{14pt}
\setlength{\parindent}{0in}

% ========== 数学相关 ==========
\usepackage{amsmath, amssymb, amsthm, bm}
\usepackage{physics} % 物理符号(如需要)
\usepackage{mathtools} % 扩展数学功能

% ========== 图形与绘图 ==========
\usepackage{graphicx}
\usepackage{tikz}
\usetikzlibrary{arrows, automata, positioning}

% ========== 表格处理 ==========
\usepackage{booktabs} % 专业表格
\usepackage{diagbox}  % 斜线表头
\usepackage{array}    % 表格列格式

% ========== 算法伪代码 ==========
\usepackage{algorithm}
\usepackage{algorithmicx}
\usepackage[noend]{algpseudocode}

% ========== 超链接与引用 ==========
\usepackage{hyperref}
\hypersetup{
    colorlinks=true,
    linkcolor=black,
    filecolor=magenta,      
    urlcolor=cyan,
}

% ========== 注释与辅助 ==========
\usepackage{comment} % 多行注释

\setlength{\headheight}{14pt}
\setlength{\parindent}{0 in}

\newtheorem{theorem}{Theorem}
\newtheorem{lemma}[theorem]{Lemma}
\newtheorem{proposition}[theorem]{Proposition}
\newtheorem{claim}[theorem]{Claim}
\newtheorem{corollary}[theorem]{Corollary}
\newtheorem{definition}[theorem]{Definition}
\newtheorem{remark}[theorem]{Remark}

\newenvironment{question}[2][Question]{\begin{trivlist}
\item[\hskip \labelsep {\bfseries #1}\hskip \labelsep {\bfseries #2.}]}{\hfill$\blacktriangleleft$\end{trivlist}}
\newenvironment{answer}[1][Answer]{\begin{trivlist}
\item[\hskip \labelsep {\bfseries #1.}\hskip \labelsep]}{\hfill$\lhd$\end{trivlist}}

\newcommand\E{\mathbb{E}}

\newcommand{\ta}[1]{{\color{red!100!teal} [{\bf TA's Note:} #1]}}

\usepackage{lastpage}
\usepackage{datetime}


% ========= 跨页代码环境 ==========
\definecolor{lightgray}{gray}{0.75}

\makeatletter
\newenvironment{algo}
  {% \begin{breakablealgorithm}
    \begin{center}
      \refstepcounter{algorithm}% New algorithm
      \hrule height.8pt depth0pt \kern2pt% \@fs@pre for \@fs@ruled
      \parskip 0pt
      \renewcommand{\caption}[2][\relax]{% Make a new \caption
        {\raggedright\textbf{\fname@algorithm~\thealgorithm} ##2\par}%
        \ifx\relax##1\relax % #1 is \relax
          \addcontentsline{loa}{algorithm}{\protect\numberline{\thealgorithm}##2}%
        \else % #1 is not \relax
          \addcontentsline{loa}{algorithm}{\protect\numberline{\thealgorithm}##1}%
        \fi
        \kern2pt\hrule\kern2pt
     }
  }
  {% \end{breakablealgorithm}
     \kern2pt\hrule\relax% \@fs@post for \@fs@ruled
   \end{center}
  }
\makeatother

\title{Homework Set \#1}
\usetikzlibrary{positioning}

\begin{document}

    \pagestyle{fancy}
    \lhead{Peking University}
    \chead{}
    \rhead{Randomized Algorithm, 2025 Spring}
    \fancyfoot[R]{} 
    \fancyfoot[C]{\thepage\ /\ \pageref{LastPage} \\ \textcolor{lightgray}{Last Compile: \today}}


    \begin{center}
        {\LARGE \bf Midterm}\\
        {Due: 2025-4-10 23:59 \quad$|$\quad 4 Questions, 100+10 Pts}\\
        {Name: Jiacong Fang, ID: 2200017849}
    \end{center}


    
    \begin{question}{1 (10') (Von Neumann Trick)}
    Given a coin with probability $p$ of getting ``head'' where $p\in(0,1)$. In this problem, we are going to use this coin to generate fair results in $\{0,1\}$. When $p=1/2$, it's quite simple. However, it may not be so trivial for general $p$. John von Neumann gave the following procedure:
    \begin{enumerate}
        \item Toss the coin twice.
        \item If two results are the same, start over, forgetting both results.
        \item If two results are different, give $0$ when the first result is ``head'', and $1$ otherwise.
    \end{enumerate}
    
    \begin{itemize}
        \item[a. (3')] Prove that, the procedure can generate uniform random results in $\{0,1\}$.
        \item[b. (7')] Calculate the expectation for the number of throws. Express the answer as a function of $p$.
    \end{itemize}
    \end{question}

\begin{answer}
    \begin{enumerate}[label = \alph*).]
        \item In each throw round, there are 3 cases:
        \begin{align*}
            \Pr[\text{results are the same}] &= p^2 + (1-p)^2 \\
            \Pr[\text{terminated, output 0}] &= \Pr[\text{terminated, output 1}] = p(1-p).  
        \end{align*}
        And for the whole procedure, we have 
        \begin{align*}
            \Pr[\text{output 0}] = \sum_{t=1}^{\infty} \left(p^2 + (1-p)^2\right)^{t-1} \cdot p(1-p) = p(1-p) \cdot \frac{1}{2p(1-p)} = \frac{1}{2}.
        \end{align*}
        Similarly, we have $\Pr[\text{output 1}] = 1/2$. Thus, the procedure can generate uniform random results in $\{0,1\}$.
        \item Denote the number of throws as $X$. Then we have 
        \begin{align*}
            \Pr(X = k) &= \left[p^2 + (1-p)^2\right]^{k-1} \cdot 2p(1-p). \\
            \E[X] &= 2\sum_{k=1}^{\infty} k \cdot 2p(1-p) \cdot \left[p^2 + (1-p)^2\right]^{k-1} \\
                &= 4p(1-p) \sum_{k=1}^{\infty} k \cdot \left[p^2 + (1-p)^2\right]^{k-1}.
        \end{align*}
        Let $g(p) = \sum_{k=0}^{\infty} (2p^2 - 2p + 1)^k = \frac{1}{2p(1-p)}$, then 
        \begin{gather*}
                g'(p) = -\frac{(1-2p)}{2p^2(1-p)^2} = \sum_{k=1}^{\infty} k(4p-2)(2p^2-2p+1)^{k-1} \\
                \implies \sum_{k=1}^{\infty} k(2p^2-2p+1)^{k-1} = \frac{1}{4p^2(1-p)^2} \implies \E[X] = \frac{4p(1-p)}{4p^2(1-p)^2} = \frac{1}{p(1-p)}.
        \end{gather*}
    \end{enumerate}
    \ed
\end{answer}
    \clearpage
    \begin{question}{2 (15') (Refined Error Bound for Schwartz--Zippel Algorithm)}
In this question, you will reflect on the error bound of Schwartz--Zippel Algorithm. 
    \begin{itemize}
        \item[a. (3')] Show the error bound in Schwartz--Zippel Algorithm is tight by giving an example. For convenience, assume $n=3$ and $S=\{1,2,3,4,5\}$. Give two distinct polynomials $P(x_1,x_2,x_3)$ and $Q(x_1,x_2,x_3)$ where the degree of $P-Q$ is $3$, such that Schwartz--Zippel Algorithm fails with probability \textit{exactly} $3/5$.
        
        \item[b. (12')] Let $f_0(x_1,x_2,\cdots,x_n)$ be a multivariate polynomial over $\mathbb R$. For each $1\leq i\leq n$, we inductively define $d_i$ to be the maximum exponent of $x_i$ in $f_{i-1}$, and $f_i(x_{i+1},\cdots,x_n)$ to be the coefficient of $x_i^{d_i}$ in $f_{i-1}$. Assume $S_1,S_2,\cdots,S_n\subseteq\mathbb R$ be arbitrary finite subsets. For $r_i\in S_i$ chosen independently and uniformly at random, show that
        $$
        \Pr\left[f_0(r_1,r_2,\cdots,r_n)=0\middle|f_0\not\equiv0\right]
        \leq\sum_{i=1}^n\frac{d_i}{|S_i|}.
        $$
    \end{itemize}
\end{question}

\begin{answer}
    \begin{enumerate}[label = \alph*).]
        \item Let $P(x_1,x_2,x_3) = (x_1-1)(x_1-2)(x_1-3) + x_1x_2x_3$ and $Q(x_1,x_2,x_3) = x_1x_2x_3$. 
        then 
        \[
            R:= P-Q = (x_1-1)(x_1-2)(x_1-3), \text{ whose degree is $3$}. 
        \]
        Therefore, the probability of failure is
        \begin{align*}
            \Pr[R(r_1,r_2,r_3) = 0 | R \not\equiv 0] &= \frac{3\cdot 5^2}{5^3} = \frac{3}{5}.
        \end{align*}
        \item Prove the statement by induction on $n$. 
        
        \textbf{Base case.} When $n=1$, $f_0$ is a single variable polynomial with at most $d_1$ roots in $S_1$. Thus, 
        \[
            \Pr[f_0(r_1) = 0 | f_0 \not\equiv 0] = \frac{d_1}{|S_1|}.
        \]
        \textbf{Induction hypothesis.} Assume the statement holds for all multivariate polynomial with at most $k-1$ variables.
        
        \textbf{Inductive step.} Consider the polynomial $f_0(x_1,x_2,\cdots,x_k)$, then we can write
        \begin{align*}
            f_0(x_1,x_2,\cdots,x_k) = f_1(x_2,\cdots,x_k)x_1^{d_1} + N(x_1, x_2, \cdots, x_k).
        \end{align*}
        Assume $r_2, r_3, \cdots, r_k$ are fixed, denote $\Omega$ as the event that $f_1(r_2, r_3, \cdots, r_k) = 0$. Then
        \begin{itemize}
            \item  When $\Omega$ happenes, by the induction hypothesis, 
            \[
                \Pr[\Omega] \leq \sum_{i=2}^k \frac{d_i}{|S_i|}
            \]
            \item When $\Omega$ does not happen, fix $r_2, r_3, \cdots, r_k$, then $f_0(r_1, r_2, \cdots, r_k)$ is a single variable polynomial with at most $d_1$ roots in $S_1$. Thus,
            \[
                \Pr[f_0(r_1, r_2, \cdots, r_k) = 0 | \neg \Omega] \le \frac{d_1}{S_1}
            \]
        \end{itemize}
        Combine the two cases, we have
        \begin{align*}
            \Pr[f_0(r_1, \cdots, r_k) = 0 | f_0 \not\equiv 0] &= \Pr[\Omega]\Pr[f_0(r_1, \cdots, r_k) = 0 | \Omega] + \Pr[\neg \Omega]\Pr[f_0(r_1,\cdots, r_k) | \neg \Omega] \\
            &\le \sum_{i=2}^k \frac{d_i}{|S_i|} + \frac{d_1}{|S_1|} = \sum_{i=1}^k \frac{d_i}{|S_i|}.
        \end{align*}
        Therefore, the statement holds for all $n \in \N^+$.
    \end{enumerate}
    \ed
\end{answer}

    \clearpage
        
    \begin{question}{3 (10') (Conditional Expectation)}~\\ 
    Let $v_1, \cdots, v_n \in \mathbb R^n$, all with $\|v_i\|_2= 1$.
    \begin{enumerate}
        \item [a. (5')] Prove that there exist $\epsilon_1, \dots, \epsilon_n =\pm 1$ such that
              $$ 
              \|\epsilon_1 v_1 + \cdots + \epsilon_n v_n\|_2 \leq \sqrt{n}.
              $$

        \item [b. (5')] Please provide a polynomial-time \textit{deterministic} algorithm for finding an assignment of $\epsilon_1,\dots, \epsilon_n$ in $\pm1$ such that $\|\epsilon_1 v_1 + \cdots + \epsilon_n v_n\|_2 \leq \sqrt{n}$, and analyze its time complexity. 

        \textit{[Hint: To derandomize the algorithm, consider how you can leverage the result from \textnormal{a.} in an inductive manner to fix the values of $\epsilon_1, \cdots, \epsilon_n$ one by one.]}
    \end{enumerate}
    \end{question}
    \clearpage
    \section*{Problem 4 (25') (Sub-Network Approximation for Two-Layer Neural Networks)} 


\begin{definition}[Two-Layer Neural Network]
    Consider a wide neural network $f_M(x; \theta):\mathbb{R}^{d}\rightarrow\mathbb{R}$ of the form

\[
f_M(x; \theta) := \sum_{j=1}^M a_j \sigma(w_j^{\top} x),
\]

where \( \theta = \{a_1, \dots, a_M, w_1, \dots, w_M\} \) denotes the parameters of the network, \( a_j \in \mathbb{R}\setminus\{0\} \) are the non-zero coefficients, and \( w_j \in \mathbb{R}^d \) are the weights associated with the \( j \)-th neuron. Here, \( \sigma:\mathbb{R}\rightarrow\mathbb{R}_{\geq 0} \) denotes ReLU activation function that is defined as $\sigma(\cdot) = \operatorname{max}\{0, \cdot\}$.
\end{definition}

\begin{definition}[Parameter Norm]
    The norm of the parameters \( \theta \) is defined as

\[
\|\theta\|_{\mathcal{P}} := \sum_{j=1}^M |a_j| \|w_j\|_2,
\]

where \( \|w_j\|_2 \) denotes the Euclidean norm of the weight vector \( w_j \), and \( \|\theta\|_{\mathcal{P}} \) is assumed to be finite.
\end{definition}
 

Let \( \pi = \mathrm{Unif}(\mathbb{S}^{d-1}) \) denote the uniform distribution over the unit sphere \( \mathbb{S}^{d-1} \subset \mathbb{R}^d \). The goal is to approximate the network \( f_M(x; \theta) \) with a sparse subset of the neurons, while controlling the error in expectation. Indeed, please prove the following result:

\begin{theorem}
    For any integer \( m \in \mathbb{N} \), there exists a sparse vector \( \tilde{a} = (\tilde{a}_1, \dots, \tilde{a}_M) \) such that \( \|\tilde{a}\|_0 = m \), i.e., the vector \( \tilde{a} \) contains exactly \( m \) non-zero entries, which satisfies the following approximation bound:

\[
\E_{x \sim \pi} \left( \sum_{j=1}^M \tilde{a}_j \sigma(w_j^{\top} x) - \sum_{j=1}^M a_j \sigma(w_j^{\top} x) \right)^2 \lesssim \frac{\|\theta\|_{\mathcal{P}}^2}{dm}.
\]
\end{theorem}


%This inequality shows that by selecting a sparse subset of \( m \) coefficients \( \tilde{a} \) from the full set of coefficients \( a_1, \dots, a_M \), we can achieve a good approximation of the network function, with the error bounded by a term proportional to the squared norm of the network parameters \( \|\theta\|_{\mathcal{P}}^2 \) divided by \( dm \), where \( d \) is the dimensionality of the input space and \( m \) is the number of selected neurons.

\begin{remark}
    In essence, this result demonstrates that it is possible to approximate the network with a sparse representation without incurring significant loss in performance, provided the number of selected neurons \( m \) is sufficiently large relative to the network's parameter norm $\|\theta\|_{\mathcal{P}}$, and the size of the input space \( d \).
\end{remark}

\begin{answer}
    We try to sample $\tilde{a}$ and construct an unbiased estimator whose expectation is $f_M(x; \theta)$.
    For $j \in [M]$, we define the following probability mass function (Denote as distribution $\mathcal{D}_M$).
    \begin{align*}
        p_j = \frac{|a_j|\cdot\norm{w_j}_2}{\norm{\theta}_{\mathcal{P}}} \quad \text{and } \sum_{j=1}^M p_j = 1.
    \end{align*}
    Then we can sample $m$ neurons from $M$ neurons with $\{p_j\}_{j=1}^M$ independently, i.e., 
    \begin{align*}
        \text{sample } j_1, j_2, \cdots, j_m \overset{i.i.d}{\sim} \mathcal{D}_M.
    \end{align*}
    Then we can define the following estimator, for $ t\in [m]$,
    \begin{align*}
        Y_t(x) := \frac{\text{sign}(a_{j_t})}{\norm{w_{j_t}}_2} \cdot \sigma(w_{j_t}^{\top} x) \quad \text{and } \tilde{f}(x) := c\cdot \sum_{t=1}^m Y_t(x).
    \end{align*}
    where $c$ is a constant waiting to be determined to makes sure that $\tilde{f}(x)$ is unbiased. Then we can assign $\tilde{a}$ as follows:
    \begin{align*}
        \tilde{a}_{i} = c\cdot \frac{\text{sign}(a_{i})}{\norm{w_{i}}_2} \text{ for } i \in \{j_t\}_{t=1}^m, \text{ and } \tilde{a}_i = 0 \text{ otherwise.}
    \end{align*}
    Notice that $j_1, j_2, \cdots, j_m$ may be not distinct, but we can merge those terms into a single nonzero entry. Thus, here we have $\norm{\tilde{a}}_0 \le m$ (it's a weaker condition than $\norm{\tilde{a}}_0 = m$ actually).

    Now we determine the constant $c$ first. Given $x$, we have:
    \begin{align*}
        \E_{j_t \sim \mathcal{D}_M}\left[Y_t(x)\right] &= \sum_{j=1}^{M} p_j \cdot \frac{\text{sign}(a_{j})}{\norm{w_{j}}_2} \sigma(w_{j}^{\top} x) \\
        & = \sum_{j=1}^{M} \frac{|a_j| \norm{w_j}_2}{\norm{\theta}_{\mathcal{P}}} \cdot \frac{\text{sign}(a_{j})}{\norm{w_{j}}_2} \sigma(w_{j}^{\top} x) \\
        &= \frac{1}{\norm{\theta}_{\mathcal{P}}} \sum_{j=1}^{M} a_j \sigma(w_{j}^{\top} x) \quad (\text{use sign}(a_j)\cdot|a_j| = a_j) \\
        &= \frac{1}{\norm{\theta}_{\mathcal{P}}} f_M(x; \theta).
    \end{align*}
    Then (let  $c = \norm{\theta}_{\mathcal{P}} / m$)
    \begin{align*}
        \E_{j_t \sim \mathcal{D}_M}\left[\tilde{f}(x)\right] &= c\cdot \sum_{t=1}^m \E_{j_t \sim \mathcal{D}_M}\left[Y_t(x)\right] = \frac{c\cdot m}{\norm{\theta}_{\mathcal{P}}} f_M(x; \theta) = f_M(x; \theta).
    \end{align*}
    Therefore, $\tilde{f}(x)$ is an unbiased estimator of $f_M(x; \theta)$ over $\mathcal{D}_M$.

    Look back the error term we want to bound, notice that:
    \begin{align*}
        &\quad \E_{x \sim \pi} \left( \sum_{j=1}^M \tilde{a}_j \sigma(w_j^{\top} x) - \sum_{j=1}^M a_j \sigma(w_j^{\top} x) \right)^2  \\
        &= \E_{x \sim \pi} \left[\left(\tilde{f}(x) - \E_{j_t \sim \mathcal{D}_M}[\tilde{f}(x)]\right)^2\right] = \E_{x \sim \pi} \left[\Var_{j_t \sim \mathcal{D}_M}\left( \tilde{f}(x) \right)\right] \\
        &= \E_{x\sim \pi} \left[\frac{\norm{\theta}_\mathcal{P}^2 \cdot m}{m^2} \Var_{j_t \sim \mathcal{D}_M}\left(Y_t(x)\right)\right] \quad \text{ ($t$ is a fixed index here)} \\
        &= \frac{\norm{\theta}_{\mathcal{P}}^2}{m} \cdot \E_{x\sim \pi} \left[\E_{j_t\sim \mathcal{D}_M}\left[\left(Y_t(x)\right)^2\right] - \left(\E_{j_t\sim \mathcal{D}_M} [Y_t(x)]\right)^2\right] \\
        &\le \frac{\norm{\theta}_{\mathcal{P}}^2}{m} \cdot \E_{x\sim \pi} \left[\E_{j_t\sim \mathcal{D}_M}\left[\left(Y_t(x)\right)^2\right]\right]  \\
        &= \frac{\norm{\theta}_{\mathcal{P}}^2}{m} \cdot  \E_{j_t\sim \mathcal{D}_M}\left[\E_{x\sim \pi}\left[\left(Y_t(x)\right)^2\right]\right].
    \end{align*}
    The inequality holds because $\E_{j_t\sim \mathcal{D}_M} [Y_t(x)]^2 = f^2_M(x; \theta) / \norm{\theta}_2^2 \ge 0$.
    Then given $j_t$, we try to calculate 
    \begin{align*}
        \E_{x\sim \pi}\left[\left(\sigma(w_{j_t}^\top x)\right)^2\right] &= \int_{\mathbb{S}^{d-1}} \left(\max\{0, w_{j_t}^\top x\}\right)^2 \cdot d\pi(x) \\
        &= \int_{\{w_{j_t}^\top x \ge 0\}} \left(w_{j_t}^\top x\right)^2 \cdot d\pi(x) \\
        &= \frac{1}{2} \int_{\mathbb{S}^{d-1}} \left(w_{j_t}^\top x\right)^2 \cdot d\pi(x) \quad \text{($w_{j_t}^\top x$ is symmetric over $\mathbb{S}^{d-1}$)} \\
        &= \frac{1}{2} \E_{x\sim \pi} \left[\left(w_{j_t}^\top x\right)^2\right] \\
        &= \frac{\norm{w_{j_t}}_2^2}{2} \cdot \E_{x\sim \pi} \left[\left(u^\top x\right)^2\right] \quad \text{(where $u = \frac{w_{j_t}}{\norm{w_{j_t}}_2}$)}.
    \end{align*}
    We have $\E_{x\sim \pi} \left[\left(u^\top x\right)^2\right] = 1/d$ (it will be proved at the end, skip it for now). Therefore,
    \begin{align*}
        \E_{x\sim \pi}\left[\left(\sigma(w_{j_t}^\top x)\right)^2\right] = \frac{\norm{w_{j_t}}_2^2}{2d}. \implies \E_{x\sim \pi}\left[Y_t(x)^2\right] = \frac{\norm{w_{j_t}}_2^2}{2d} \cdot \frac{\text{sign}(a_{j_t})^2}{\norm{w_{j_t}}_2^2} = \frac{1}{2d}.
    \end{align*}
    Then we have
    \begin{align*}
        \E_{x\sim \pi} \left( \sum_{j=1}^M \tilde{a}_j \sigma(w_j^{\top} x) - \sum_{j=1}^M a_j \sigma(w_j^{\top} x) \right)^2 &\le \frac{\norm{\theta}_{\mathcal{P}}^2}{m} \cdot  \E_{j_t\sim \mathcal{D}_M}\left[\frac{1}{2d}\right] = \frac{\norm{\theta}_{\mathcal{P}}^2}{2dm} \lesssim \frac{\norm{\theta}_{\mathcal{P}}^2}{dm}.
    \end{align*}
    At the end, we prove that 
    \[
        \E_{x\sim \pi} \left[\left(u^\top x\right)^2\right] = 1/d.
    \]
    Expand $u$ to $d$-orthonormal basis of $\mathbb{R}^d$, denoted as $\{e_1, e_2, \cdots, e_d\}$, then for any given $x \in \mathbb{S}^{d-1}$, we have
    \begin{gather*}
        x = \sum_{i=1}^d \alpha_i e_i \quad \text{where } \sum_{i=1}^d \alpha_i^2 = 1, \implies \sum_{i=1}^{d} (e_i^\top u)^2 = \sum_{i=1}^{d} \alpha_i^2 = 1 \\
        \implies 1 = \E_{x\sim \pi} \left[\sum_{i=1}^{d} (e_i^\top u)^2 \right] = d \cdot \E_{x\sim \pi} \left[\left(u^\top x\right)^2\right]  \implies \E_{x\sim \pi} \left[\left(u^\top x\right)^2\right] = \frac{1}{d}.
    \end{gather*}
    \ed
\end{answer}
    \clearpage 
    
\end{document}